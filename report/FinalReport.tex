\documentclass[11pt, oneside]{article} 
\usepackage[margin=3.4cm]{geometry} 
\geometry{letterpaper}
\usepackage{graphicx}
\usepackage{hyperref}
\usepackage{amssymb}
\usepackage{caption}
\usepackage{float}
\usepackage{setspace}
\usepackage{esvect}
\usepackage{url}
\usepackage{lipsum} 

%\setcounter{secnumdepth}{0}
\onehalfspacing

\newcounter{refno}
\newcommand{\reflabel}[1]{\refstepcounter{refno}\label{#1}[\arabic{refno}]}

\begin{document}
% Title Page
\begin{titlepage}
	\newcommand{\HRule}{\rule{\linewidth}{0.5mm}}
	
	\center
	\textsc{\LARGE University of Calgary}\\[1.5cm]
	\textsc{\Large ENGO 651: Advanced Geospatial Topics}\\[0.5cm]
	\textsc{\large Final Project}\\[0.5cm]

	\HRule\\[0.4cm]
	{\huge\bfseries Fuellytics: Real-Time Vehicular Fuel Consumption and Emissions Monitoring}\\[0.4cm]
	\HRule\\[1.5cm]
	
	{\large\textit{Group \#}}\\
	Adam \textsc{Smith} (30031453)\\
	Chavisa \textsc{Sornsakul} (123456789)\\
	Wai Ka \textsc{Wong Situ} (123456789)\\
	
	\vfill\vfill\vfill
	{\large \today}
	
	\vfill\vfill\vfill
	\includegraphics[width=8cm,]{img/schulich.png}\\[1cm]	
\end{titlepage}

\section{Executive summary}

\newpage

\section{Problem statement}
According to the International Energy Agency, transportation accounts for almost one-quarter of global greenhouse gas emissions, and within that, road transport is responsible for the largest share of emissions \reflabel{transport-emissions}.  The burning of fossil fuels in vehicles produces carbon dioxide, which is the most prevalent greenhouse gas contributing to global warming.  Although electric vehicles are becoming more popular in the developed world, still 91\% of all transport relies on oil-based products such as gasoline, which is only a 3\% drop from the early 1970's \reflabel{oil-based-transport}.  Reducing carbon emissions from vehicles is crucial to mitigate the harmful effects of climate change and reduce the environmental burden of vehicular transport.

Tracking vehicle fuel consumption is essential to understanding the amount of carbon emissions produced by vehicles. Fuel consumption data can provide valuable insights into the efficiency of a vehicle and its impact on the environment, and can also provide insights to the driver about their driving style and fuel costs.  However, fuel consumption is heavily dependent on various factors such as vehicle type, terrain, and driving style, making it difficult to measure over short time periods.  According to the US Department of Energy, obeying speed limits, accelerating and braking gently, and reading the road ahead can improve fuel economy by up to 30\% on highways and up to 40\% in stop-and-go traffic \reflabel{driving-style}.  Moreover, engine size, vehicle weight, and driving in hilly or mountainous areas can drastically increase fuel consumption and, consequently, carbon emissions.  Some modern vehicles are equipped with a dashboard gauge that displays fuel consumption while driving, however, these gauges are often vague and uninformative, usually showing nothing more than an unlabelled bar or dial which increases while accelerating.  Without a means to directly monitor fuel consumption, drivers are often left unaware of their fuel use and how their driving style and driving conditions can reduce their vehicle's impact on the environment.

The presented smartphone application, \textit{Fuellytics}, provides an innovative solution to monitoring fuel consumption in real-time, and provides insights and reports to drivers about their fuel use while driving as well as on a weekly and monthly basis.  Fuellytics helps drivers better understand the environmental impacts of their vehicle, make educated decisions regarding transportation, and save money on fuel.

\section{Similar solutions and available literature}

For industrial applications, several fleet management solutions exist that offer analytics on fuel consumption and tracking.  For example, \textit{FuelForce Fuel Management Systems} \reflabel{fuel-force} provides tools to monitor, track, view, and analyze fuel use of fleet vehicles, and \textit{Triscan} \reflabel{triscan} provides a similar functionality by integrating their systems with refuelling stations. These solutions, however, are targeted at industrial applications, and do not provide real-time analytics to the driver.  Moreover, these and similar tools are focused around fleet management and cost savings, and often do not provide an analysis of environmental impact.

Multiple applications for personal use exist which track fuel usage over long periods of time to determine fuel costs and fuel economy. The web and iOS application \textit{Fuelly} \reflabel{fuelly} allows users to enter volume and cost data each time they refuel to gain insight into their fuel consumption and fuel costs over time.  A similar application, \textit{Fuel Line} \reflabel{fuelline} was recently developed by a team at BCIT and additionally contains a routing feature to predict fuel costs on future routes.  Notably, these solutions differ from Fuelltyics in that they do not provide real-time fuel consumption data while driving and they are focused more on monitoring costs rather than reducing environmental impacts.

Some applications developed for carbon emission tracking provide functionality to track CO$_2$ emission in many categories including travel.  Examples of such applications include \textit{MyEarth} \reflabel{myearth} and \textit{Carbon Footprint \& CO2 Tracker} \reflabel{capture}.  However, the main focus of these applications is tracking carbon emissions in a broad range of categories, not an in-depth and real-time analysis of carbon emissions from driving.

Now we talk about the papers on the subject...


\vspace{1cm}


Papers on the subject: Main reference: \url{https://www.mdpi.com/2076-3417/9/7/1369#B21-applsci-09-01369}

\section{Solution summary}

\section{Architecture}

\subsection{Design rationale}

\subsection{Architecture description}

\subsection{API}

\subsection{Sequence Diagram}

\subsection{Data models and JSON encodings}

\section{Results}

\section{Lessons learned}

\section{Conclusion}
 
 
\section{References}

\begin{small}

\noindent [\ref{transport-emissions}] \url{https://www.iea.org/reports/transport-energy-and-co2}

\noindent [\ref{oil-based-transport}] \url{https://www.iea.org/reports/transport}

\noindent [\ref{driving-style}] \url{https://afdc.energy.gov/conserve/behavior_techniques.html}

\noindent [\ref{fuel-force}] \url{https://info.gartnerdigitalmarkets.com/multiforce-gdm-lp?channel=capterra}

\noindent [\ref{triscan}] \url{https://www.thetriscangroup.com/}

\noindent [\ref{fuelly}] \url{https://www.fuelly.com}

\noindent [\ref{fuelline}] \url{https://fuel-line.fly.dev/}

\noindent [\ref{myearth}] \url{https://play.google.com/store/apps/details?id=com.ionicframework.myearthapp496614&hl=en_CA&gl=US}

\noindent [\ref{capture}] \url{https://play.google.com/store/apps/dev?id=7417037584465272817&hl=en_CA&gl=US}



\end{small}

\end{document}